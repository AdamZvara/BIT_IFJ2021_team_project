\documentclass[11pt]{article}
\usepackage[utf8]{inputenc}
\usepackage[czech]{babel}
\usepackage[left=2cm,text={17cm, 24cm},top=3cm,a4paper]{geometry}
\usepackage{times}
\usepackage{csquotes}
\usepackage{graphics}
\usepackage{float}
\usepackage{multirow}
\usepackage{boldline}

\begin{document}

\begin{titlepage}
\begin{center}

\Huge\textsc{Fakulta informačních technologií\\
Vysoké učení technické v Brně}\\
\vspace{\stretch{0.382}}
\LARGE
\textbf{Překladač jazyka IFJ21\,-\,Projektová dokumentace}\\
Tým 004\,-\,varianta II.
\vspace{\stretch{0.100}}

\begin{table}[ht]
    \centering
    \begin{tabular}{l l l}
        \textbf{Vojtěch Eichler} & \textbf{xeichl01} & \textbf{25\,\%}\\
        Adam Zvara & xzvara01 & 25\,\%\\
        Václav Korvas & xkorva03 & 25\,\%\\
        Tomáš Matuš & xmatus37 & 25\,\%
    \end{tabular}
\end{table}
\vspace{\stretch{0.518}}

\Large
2021
\end{center}
\end{titlepage}

\section{Úvod}
Cílem tohoto projektu bylo vytvořit překladač, který přeloží zadaný jazyk IFJ21 do cílového mezijazyka \mbox{IFJCode2021.}
Jazyk IFJ21 je staticky typovaný a~imperativní jazyk. Jedná se o podmožinu jazyka Teal.
Překladač je napsaný v jazyce C.

\section{Implementace}
Překladač je sestaven ze 3 hlavních částí a to scanner, parser a generátor kódu.
Scanner provádí lexikální analýzu a načítání vstupního kódu, parser pak provádí syntaktickou a sémantickou analýzu.

\subsection{Lexikální analýza}
Lexikální analýza je implementována v souborech \texttt{scanner.c} a \texttt{scanner.h}. Hlavní funkce lexikálního analyzátoru je funkce \texttt{get\_token},
která postupně čte znak po znaku ze standardního vstupu. Tyto znaky poté převádní na strukturu \texttt{token\_t}. Tato struktura obsahuje atribut a typ tokenu.
Jako typ tokenu může být identifikátor, řetězec, celé nebo desetinné číslo, EOF, aritmetické a porovnávací operátory nebo jiné znaky povolené v jazyce IFJ21.
Atribut mají pouze tokeny typu řetězec, identifikátor, klíčové slovo nebo čísla. Zda-li se jedná o~klíčové slovo nebo identifikátor
se stará pomocná funkce \texttt{check\_keyword}, která porovná danný řetězec se všemi klíčovými slovami, pokud se s žádným z nich neshoduje nastaví jako typ
tokenu identifikátor. Funkce \texttt{get\_token} pak také využívá řadu dalších pomocných funkcí například pro převod řetězce na číslo.

Lexikální analyzátor byl implementován na základě námi vytvořeného deterministického konečného automatu (DKA).
Tento DKA je implementován jako jeden nekonečně se opakující switch, kde jednotlivé přípády case reprezentují jednotlivé stavy tohoto automatu.
Pokud se automat dostane do jednoho z koncových stavů, tak se funkce ukončí a vrátí odpovídající token. Pokud ovšem
automat skončí v nekoncovém stavu nebo je načten znak, který není povolen v jazyce IFJ21, tak nastává lexikální chyba a funkce je ukončena s návratovým kódem 1.

Escape sekvence jsou řešeny za pomoci pole, do kterého se postupně ukládají znaky, které mohou být v rozsahu 0-9. Toto decimalní číslo jehož maximální
hodnota je 255, je poté převedeno na korespondující ASCII hodnotu.

\subsection{Syntaktická analýza}
Syntaktická analýza je implementována metodou rekurzivního sestupu. Překlad zdrojového kódu začíná zavolaním funkce
\texttt{parse} z funkce main ve zdrojovém souboru \texttt{parser.c}. Následuje inicializace tokenu, do kterého lexikální analyzátor
zapisuje hodnoty, globální tabulky symbolů, paměti pro postupné ukládání generovaných instrukcí a také se nainicializuje pomocná struktura.

Během implementace jsme postupně zjišťovali,
že bychom pro generování a sémentické kontroly potřebovali držet v paměti různé hodnoty jako například počty parametrů u volání funkce,
ukazatel na funkci do globální tabulky symbolů a několik dalších hodnot. Z tohoto důvodu jsme vytvořili pomocnou strukturu
\texttt{parser\_helper} a funkce nad ní, implementovanou v souboru \texttt{parser\_helper.c}.

 Z funkce \texttt{parse} se
 syntaktický analyzátor postupně rekurzivně zanořuje podle sekvence tokenů, které získavá od lexikálního analyzátoru. Po celou
 dobu běhu se v globální proměnné \texttt{ret} udržuje aktuální stav a pokud analyzátor objeví chybu, nebo nastane interní chyba,
 tato proměnná se nastaví na nenulovou hodnotu a začne vynořování z rekurze.

 Jedna z největších výzev byla absence ukončovacího
 znaku za příkazem, nebo pevně stanovené pravidla odsazování, což zkomplikovalo práci s načtenými tokeny. Proto analyzátor pracuje
 s proměnnou \texttt{backup\_token}, která kromě role dočasného držení tokenu, plní roli při rozhodování, zda je nutné načítat
 další token, nebo je potřeba pracovat s tokenem který už má načtený. Tímto způsobem jsme řešili situaci, kdy je při
 vyhodnocování výrazů nevyhnutelně načten jeden token za koncem výrazu, jinak totiž nejsme schopní určit konec výrazu.



\subsubsection{Syntaktická analýza výrazů}
Analýza výzazů se provadí metodou zdola nahoru.
K tomuto bylo zapotřebí sestavit precedenční tabulku. Pro její zjednodušení jsme sloučili operátory
se stejnou asociativitou a prioritou. Celá analýza výrazů je implementována v souboru \texttt{expression.c}
a využívá stack implementovaný ve \texttt{stack.c}.

Použitý algoritmus vychází z algoritmu představeného na přednášce IFJ.
Po zavolání funkce \texttt{expression} se inicializuje stack a vloží se na něj pomocný znak \texttt{DOLLAR},
který značí začátek výrazu. Načtený token se funkcí \texttt{token\_to\_symbol} převeda na vnitřní symbol,
se kterým se dále pracuje na stacku. Pomocí funkce \texttt{symbol\_to\_index} se zjistí z vrcholu stacku
a nového symbolu souřadnice precedenčního znaku tabulce.

\begin{itemize}
    \item Pro znak $=$ se uloží na zásobník načtený symbol a provede se načtení nového tokenu.
    \item Pro znak $<$ se uloží na zásobník pomocný znak \texttt{HANDLE}, symbol načtený ze vstupu
        a provede se načtení dalšího tokenu.
    \item Pro znak $>$ se provede funkce \texttt{reduce},
        která podle daných pravidel redukuje symboly uložené na zásobníku.
        Funkce zjistí kolik symbolů se nachází před \texttt{HANDLE}a podle jejich typu deterministicky volí
        redukční pravidlo. Pravidlo se provede odebráním potřebného počty symbolů ze zásobníku
        a nahradíme je symbolem \texttt{NON\_TERM}.
        Pokud neexistuje pravidlo pro redukci, tak funkce hlásí syntaktickou chybu.
\end{itemize}

Algoritmus se provádí dokud se nenarazí na ukončení. Nakonec se zkontroluje stav stacku a pokud se liší od \$E,
funkce hlásí syntaktickou chybu.

Dále se zde řeší přiřazení funkce, tzn. pokud narazí na identifikátor, který se nachází v globální tabulce,
funkce končí a předává řízení zpět do parseru.

\subsubsection{Stack pro syntaktickou analýzu}
Pro precedenční analýzu se využívá stack, na který se ukládají vstupní symboly a provádí se na něm redukování
výrazů podle pravidel.

Implementace se nachází ve \texttt{stack.c}. Umožňuje všechny klasické operace nad zásobníkem a navíc
umožňuje několik funkcí navíc pro zjednodušení kontroly pravidel. Funkce \texttt{push\_above\_term} slouží pro
vložení symbolu za první terminál na stacku. Funkce \texttt{items\_to\_handle} vrací počet symbolů,
které se nachází před \texttt{HANDLE} a využívá se pro vyhodnocení typu pravidla.
Funkce \texttt{find\_len\_op} vyhledá na stacku operátor \# a funkce \texttt{get\_top\_operator} vrací operátor
na vrcholu zásobníku. Obě funkce se používájí pro sémantickou kontrolu.

\subsection{Sémantická analýza}
 Sémantická analýza pracuje se dvěma typy tabulek symbolů. První z nich je globální tabulka symbolů, do které jsou na začátku
 syntaktické analýzy uloženy vestavěné funkce a postupně během překladu i uživatelem definované funkce. Druhý typ tabulky je lokální
 tabulka symbolů, na začátku funkce se vytvoří první lokální tabulka a při dalším zanoření do cyklu, nebo v těle podmínky se vytvoří
 další, která se pomocí ukazatele napojí na ostatní lokální tabulky. Vznikne tak jednostranně vázaný seznam, na jehož začátku se nachází
 nejvíce zanořená tabulka a končí tabulkou vytvořenou při vstupu do těla funkce. Průchodem seznamem lokálních tabulek jsme tak
 schopni kontrolovat zda a případně v jakém bloku platnosti jsou proměnné definované. Obě tabulky jsou naimplementované v souboru
 \texttt{symtable.c}.

Sémantická analýza ke své činnosti dále využívá řetězce na ukládání typů různých hodnot a strukturu \texttt{parser\_helper}, popsanou v kapitole o syntaktické analýze. 
Do řetězců se ukládají například typy proměnných při inicializaci, které se pak porovnávají s hlavičkou funkce, nebo s typy hodnot, 
kterými dané proměnné inicializujeme. Tímto způsobem jsme schopni kontrolovat nejen typy hodnot, ale i jejich počet, jelikož se do řetězce ukládajá 
jen jeden znak pro každou hodnotu. Obdobně jsou řešené i parametry u volání funkce, nebo například návratové hodnoty u funkcí.
\subsection{Generování kódu}
Pro generování kódu jsme zvolili metodu přímého generování zásobníkových instrukcí během rekurzivního sestupu v syntaktické analýze.

Instrukce se generují na základě aktuálně vykonávané funkce v syntaktické analýze a také na základě kontextu (informace uložené v struktuře
\texttt{parser\_helper}). Vygenerované instrukce se ukládají do pomocné struktury \texttt{ibuffer} - struktura definována v souboru \texttt{ibuffer.c}
obsahující "flexible array member", do kterého se ukládají samotné instrukce, které se v průběhu parsování tisknou na standardní výstup.

Jeden z problémů je generovaní unikátních jmen identifikátorů mezi funkcemi nebo při vytvoření nového lokálního rámce ve funkci (podmínka\,/\,cyklus).
Tento problém jsme vyřešili pomocí generovaní unikátního jména identifikátoru: \texttt{jméno\_funkce\$zanoření\$jméno\_identifikátoru} pro proměnné,
které jsou v hlavním těle funkce. Jména proměnných v nových lokálních rámcích jsou navíc doplněna o \texttt{if\_counter} a \texttt{while\_counter},
které představují počet aktuálních zanoření, aby se dala vytvářet stejná jména proměnných uvnitř více lokálních rámců. Tedy výsledný formát je:

\begin{quotation}
    \texttt{jméno\_funkce\$zanoření\$if\_counter\$while\_counter\$jméno\_identifikátoru}
\end{quotation}

Další problém, na který jsme při generovaní narazili, je deklarace identifikátorů uvnitř cyklů. To jsme vyřešili použitím pomocné struktury
\texttt{defvar\_buffer} a stavového řetězce v struktuře \texttt{parser\_helper}. Pokud stavový řetězec obsahuje informaci, že jsme v těle cyklu,
všechny deklarace identifikátorů se generují do \texttt{defvar\_bufferu} (ostatní instrukce se generují do \texttt{ibufferu}). Když zjistíme,
že se nenacházíme v cyklu, vytiskne se obsah \texttt{defvar\_bufferu} a následně obsah \texttt{ibufferu}, čímž zabráníme vícenásobné deklaraci proměnné.

Vestavěné funkce se generují na konci celého programu. Aby jsme negenerovali zbytečně všechny vestavěné funkce, používáme strukturu \texttt{builtin\_used},
do které si ukládáme informace o použitých funkcích, které se pak generují. Speciálním případem je funkce \texttt{write}, která se generuje přímo na místě volání této funkce.

\section{Práce v týmu}

\subsection{Rozdělení práce}

\begin{table}[H]
    \centering
    \begin{tabular}{|l|l|}
        \hline
        \textbf{Člen týmu} & \textbf{Přidělená práce} \\\hline
        Vojtěch Eichler & Vedoucí týmu, syntaktická analýza, sémantická analýza, testování \\
        Adam Zvara & Sémantická analýza, generování kódu, testování \\
        Václav Korvas & Lexikální analýza, testování \\
        Tomáš Matuš & Precedenční analýza výrazů, sémantická analýza, dokumentace \\\hline
    \end{tabular}
    \label{tabulka_rozdeleni_prace}
    \caption{Rozdělení práce v týmu}
\end{table}

\subsection{Komunikace}
Našim hlavním komunikačním nástrojem byl Discord.
Dále jsme se osobně setkávali na předem dohodnutých schůzkách ve studovnách a nebo se bavili během přestávek mezi přednáškami.

\subsection{Verzovací systém}
Pro verzování jsme používali nástroj \texttt{git} a pro vzdálené sdílení repozitáře jsme využili \texttt{GitHub}.

\section{Diagram konečného automatu}
\begin{figure}[H]
    \centering
    \vspace{0.1cm}
    \scalebox{0.5}{\includegraphics{doc/img/FSM.png}}
    \caption{Diagram konečného automatu specifikující lexikální analyzátor}
    \label{figure:fsm_img}
\end{figure}

\begin{table}[]
    \centering
    \begin{tabular}{|ll ll ll|}
        \hline
        F1 & \texttt{NOT\_EQUAL} & F14 & \texttt{LEFT\_BRACKET} & Q2 & \texttt{CONCAT\_START} \\
        F2 & \texttt{GREATER\_THAN} & F15 & \texttt{RIGHT\_BRACKET} & Q3 & \texttt{STRING\_START} \\
        F3 & \texttt{GREATER\_THAN\_EQUAL} & F16 & \texttt{STRING\_LITERAL} & Q4 & \texttt{STRING\_ESCAPE} \\
        F4 & \texttt{LESSER\_THAN} & F17 & \texttt{INTEGER} & Q5 & \texttt{STRING\_ESCAPE\_HEXADEC\_1} \\
        F5 & \texttt{LESSER\_THAN\_EQUAL} & F18 & \texttt{NUMBER} & Q6 & \texttt{STRING\_ESCAPE\_HEXADEC\_2} \\
        F6 & \texttt{ASSIGN} & F19 & \texttt{MINUS} & Q7 & \texttt{NUMBER\_START\_DOT} \\
        F7 & \texttt{IS\_EQUAL} & F20 & \texttt{COMMENT} & Q8 & \texttt{NUMBER\_E} \\
        F8 & \texttt{PLUS} & F21 & \texttt{EOL} & Q9 & \texttt{NUMBER\_E\_PLUS\_MINUS}\\
        F9 & \texttt{MUL} & F22 & \texttt{EOF} & Q10 & \texttt{COMMENT\_START} \\
        F10 & \texttt{DIV} & F23 & \texttt{COMMA} & Q11 & \texttt{COMMENT\_SKIP} \\
        F11 & \texttt{INT\_DIV} & F24 & \texttt{STRING\_LENGTH} & Q12 & \texttt{COMMENT\_BLOCK\_START} \\
        F12 & \texttt{ID} & F25 & \texttt{COMMA} & Q13 & \texttt{COMMENT\_BLOCK} \\
        F13 & \texttt{CONCAT} & Q1 & \texttt{NOT\_EQUAL\_START} & Q14 & \texttt{COMMENT\_BLOCK\_STOP} \\
        \hline
    \end{tabular}
    \caption{Legenda konečného automatu pro lexikální analýzu}
    \label{tab:FSM legenda}
\end{table}

\section{LL\,-\,gramatika}

\section{LL\,-\,tabulka}

\section{Precedenční taulka}
\begin{table}[h]
    \catcode`\-=12
    \centering
    \newcolumntype{?}{!{\vrule width 1pt}}
    \begin{tabular}{?c?c|c|c|c|c|c|c|c|c|c?}
        \hlineB{3}
        \multicolumn{11}{?c?}{\textbf{Načtený token}} \\\hlineB{3}
        \parbox[t]{2mm}{\multirow{10}{*}{\rotatebox{90}{\textbf{Vrchol stacku}}}}
        &&\# & * / // & + - & .. & r & ( & ) & id & \$                       \\\cline{2-11}
        &\#      & $-$ & $>$ & $>$ & $-$ & $>$ & $<$ & $-$ & $<$ & $>$       \\\cline{2-11}
        &* / //  & $<$ & $>$ & $>$ & $-$ & $>$ & $<$ & $>$ & $<$ & $>$       \\\cline{2-11}
        &+ -     & $<$ & $<$ & $>$ & $-$ & $>$ & $<$ & $>$ & $<$ & $>$       \\\cline{2-11}
        &..      & $-$ & $-$ & $-$ & $<$ & $>$ & $<$ & $>$ & $<$ & $>$       \\\cline{2-11}
        &r       & $<$ & $<$ & $<$ & $<$ & $-$ & $<$ & $>$ & $<$ & $>$       \\\cline{2-11}
        &(       & $<$ & $<$ & $<$ & $<$ & $<$ & $<$ & $=$ & $<$ & $-$       \\\cline{2-11}
        &)       & $>$ & $>$ & $>$ & $>$ & $>$ & $-$ & $>$ & $>$ & $>$       \\\cline{2-11}
        &id      & $-$ & $>$ & $>$ & $>$ & $>$ & $-$ & $>$ & $>$ & $>$       \\\cline{2-11}
        &\$      & $<$ & $<$ & $<$ & $<$ & $<$ & $<$ & $-$ & $<$ & $-$       \\\hlineB{3}
    \end{tabular}
    \caption{Precedenční tabulka}
    \label{tab:prec_table}
\end{table}

\end{document}
