\documentclass[11pt]{article}
\usepackage[utf8]{inputenc}
\usepackage[czech]{babel}
\usepackage[left=2cm,text={17cm, 24cm},top=3cm,a4paper]{geometry}
\usepackage{times}
\usepackage{csquotes}
\usepackage{graphics}
\usepackage{float}
\usepackage{multirow}
\usepackage{boldline}

\begin{document}

\begin{titlepage}
\begin{center}

\Huge\textsc{Fakulta informačních technologií\\
Vysoké učení technické v Brně}\\
\vspace{\stretch{0.382}}
\LARGE
\textbf{Překladač jazyka IFJ21\,-\,Projektová dokumentace}\\
Tým 004\,-\,varianta II.
\vspace{\stretch{0.100}}

\begin{table}[ht]
    \centering
    \begin{tabular}{l l l}
        \textbf{Vojtěch Eichler} & \textbf{xeichl01} & \textbf{25\,\%}\\
        Adam Zvara & xzvara01 & 25\,\%\\
        Václav Korvas & xkorva03 & 25\,\%\\
        Tomáš Matuš & xmatus37 & 25\,\%
    \end{tabular}
\end{table}
\vspace{\stretch{0.518}}

\Large
2021
\end{center}
\end{titlepage}

\section{Úvod}
Cílem tohoto projektu bylo vytvořit překladač, který přeloží zadaný jazyk IFJ21 do cílového mezijazyka \mbox{IFJCode2021.}
Jazyk IFJ21 je staticky typovaný a~imperativní jazyk. Jedná se o podmožinu jazyka Teal.
Překladač je napsaný v jazyce C.

\section{Implementace}
Překladač je sestaven ze 3 hlavních částí a to scanner, parser a generátor kódu.
Scanner provádí lexikální analýzu a načítání vstupního kódu, parser pak provádí syntaktickou a sémantickou analýzu.

\subsection{Lexikální analýza}

\subsection{Syntaktická analýza}

\subsubsection{Syntaktická analýza výrazů}

\subsection{Sémantická analýza}

\subsection{Generování kódu}

\section{Práce v týmu}

\subsection{Rozdělení práce}

\begin{table}[ht]
    \centering
    \begin{tabular}{|l|l|}
        \hline
        \textbf{Člen týmu} & \textbf{Přidělená práce} \\\hline
        Vojtěch Eichler & Vedoucí týmu, syntaktická analýza, sémantická analýza, testování \\
        Adam Zvara & Sémantická analýza, generování kódu, testování \\
        Václav Korvas & Lexikální analýza, testování \\
        Tomáš Matuš & Precedenční analýza výrazů, sémantická analýza, dokumentace \\\hline
    \end{tabular}
    \label{tabulka_rozdeleni_prace}
    \caption{Rozdělení práce v týmu}
\end{table}

\subsection{Komunikace}
Našim hlavním komunikačním nástrojem byl Discord.
Dále jsme se osobně setkávali na předem dohodnutých schůzkách ve studovnách a nebo se bavili během přestávek mezi přednáškami.

\subsection{Verzovací systém}
Pro verzování jsme používali nástroj \texttt{git} a pro vzdálené sdílení repozitáře jsme využili \texttt{GitHub}.

\section{Diagram konečného automatu}
\begin{figure}[H]
    \centering
    \vspace{0.1cm}
    \scalebox{0.5}{\includegraphics{doc/img/FSM.png}}
    \caption{Diagram konečného automatu specifikující lexikální analyzátor}
    \label{figure:fsm_img}
\end{figure}

\begin{table}[]
    \centering
    \begin{tabular}{|ll ll ll|}
        \hline
        F1 & \texttt{NOT\_EQUAL} & F14 & \texttt{LEFT\_BRACKET} & Q2 & \texttt{CONCAT\_START} \\
        F2 & \texttt{GREATER\_THAN} & F15 & \texttt{RIGHT\_BRACKET} & Q3 & \texttt{STRING\_START} \\
        F3 & \texttt{GREATER\_THAN\_EQUAL} & F16 & \texttt{STRING\_LITERAL} & Q4 & \texttt{STRING\_ESCAPE} \\
        F4 & \texttt{LESSER\_THAN} & F17 & \texttt{INTEGER} & Q5 & \texttt{STRING\_ESCAPE\_HEXADEC\_1} \\
        F5 & \texttt{LESSER\_THAN\_EQUAL} & F18 & \texttt{NUMBER} & Q6 & \texttt{STRING\_ESCAPE\_HEXADEC\_2} \\
        F6 & \texttt{ASSIGN} & F19 & \texttt{MINUS} & Q7 & \texttt{NUMBER\_START\_DOT} \\
        F7 & \texttt{IS\_EQUAL} & F20 & \texttt{COMMENT} & Q8 & \texttt{NUMBER\_E} \\
        F8 & \texttt{PLUS} & F21 & \texttt{EOL} & Q9 & \texttt{NUMBER\_E\_PLUS\_MINUS}\\
        F9 & \texttt{MUL} & F22 & \texttt{EOF} & Q10 & \texttt{COMMENT\_START} \\
        F10 & \texttt{DIV} & F23 & \texttt{COMMA} & Q11 & \texttt{COMMENT\_SKIP} \\
        F11 & \texttt{INT\_DIV} & F24 & \texttt{STRING\_LENGTH} & Q12 & \texttt{COMMENT\_BLOCK\_START} \\
        F12 & \texttt{ID} & F25 & \texttt{COMMA} & Q13 & \texttt{COMMENT\_BLOCK} \\
        F13 & \texttt{CONCAT} & Q1 & \texttt{NOT\_EQUAL\_START} & Q14 & \texttt{COMMENT\_BLOCK\_STOP} \\
        \hline
    \end{tabular}
    \caption{Legenda konečného automatu pro lexikální analýzu}
    \label{tab:FSM legenda}
\end{table}

\section{LL\,-\,gramatika}

\section{LL\,-\,tabulka}

\section{Precedenční taulka}
\begin{table}[h]
    \catcode`\-=12
    \centering
    \newcolumntype{?}{!{\vrule width 1pt}}
    \begin{tabular}{?c?c|c|c|c|c|c|c|c|c|c?}
        \hlineB{3}
        \multicolumn{11}{?c?}{\textbf{Načtený token}} \\\hlineB{3}
        \parbox[t]{2mm}{\multirow{10}{*}{\rotatebox{90}{\textbf{Vrchol stacku}}}}
        &&\# & * / // & + - & .. & r & ( & ) & id & \$                       \\\cline{2-11}
        &\#      & $-$ & $>$ & $>$ & $-$ & $>$ & $<$ & $-$ & $<$ & $>$       \\\cline{2-11}
        &* / //  & $<$ & $>$ & $>$ & $-$ & $>$ & $<$ & $>$ & $<$ & $>$       \\\cline{2-11}
        &+ -     & $<$ & $<$ & $>$ & $-$ & $>$ & $<$ & $>$ & $<$ & $>$       \\\cline{2-11}
        &..      & $-$ & $-$ & $-$ & $<$ & $>$ & $<$ & $>$ & $<$ & $>$       \\\cline{2-11}
        &r       & $<$ & $<$ & $<$ & $<$ & $-$ & $<$ & $>$ & $<$ & $>$       \\\cline{2-11}
        &(       & $<$ & $<$ & $<$ & $<$ & $<$ & $<$ & $=$ & $<$ & $-$       \\\cline{2-11}
        &)       & $>$ & $>$ & $>$ & $>$ & $>$ & $-$ & $>$ & $>$ & $>$       \\\cline{2-11}
        &id      & $-$ & $>$ & $>$ & $>$ & $>$ & $-$ & $>$ & $>$ & $>$       \\\cline{2-11}
        &\$      & $<$ & $<$ & $<$ & $<$ & $<$ & $<$ & $-$ & $<$ & $-$       \\\hlineB{3}
    \end{tabular}
    \caption{Precedenční tabulka}
    \label{tab:prec_table}
\end{table}

\end{document}
